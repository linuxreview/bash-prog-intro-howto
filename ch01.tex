\chapter{معرفی}
این مقاله‌ی آموزشی می‌خواهد به شما در شروع برنامه‌نویسی خطِ فرمان کمک کند‌. این
یک آموزش سطح بالا نیست (‌به عنوان کتاب دقت کنید‌)‌ و من یک برنامه‌نویس حرفه‌ایِ
خط فرمان نیستم‌! تصمیم گرفتم که این آموزش را بنویسم‌، چون هم خیلی چیز‌ها را خودم
یاد خواهم گرفت و هم فکر کردم که این مقاله می‌تواند برای دیگران هم مفید باشد‌. از
هر باز‌خوردی قدر‌دانی خواهد شد‌، مخصوصا از اصلاحات.

دریافت آخرین نسخه کتاب اصلی:

\begin{flushleft}
\url{http://www.linuxdoc.org/HOWTO/Bash-Prog-Intro-HOWTO.html}
\end{flushleft}

دریافت آخرین نسخه کتاب ترجمه شده:

\begin{flushleft}
\url{http://ur1.ca/9djpg}
\end{flushleft}



\section*{پیش‌نیازها}
آشنایی با خط فرمان گنو‌/‌لینوکس‌، و آشنایی با نکات پایه‌ای برنامه‌نویسی می‌توانند
به مخاطب کمک کنند. به هر حال هدف این نوشته آموزش برنامه‌نویسی نیست‌‌، این کتاب قصد
آموزش نکات پایه‌ای برنامه‌نویسی
\lr{BASH}
 را دارد‌، تا چه مقبول افتد و چه در نظر آید.

\section*{کاربرد‌های این کتاب}
این کتاب تلاش می‌کند در سناریوهای زیر مفید باشد‌:

\begin{itemize}
\item{
		شما یک ایده‌ی برنامه‌نویسی دارید‌ و می‌خواهید کمی برنامه‌نویسی خط فرمان را تجربه
		کنید‌.
}

\item{
شما ابهاماتی در مورد برنامه‌نویسی خط فرمان دارید و به دنبال یک مرجع طبقه‌بندی شده
هستید‌.
}

\item{
می‌خواهید چند اسکریپت خط فرمان و کامنت
\footnote{
کامنت، توضیحاتی است که برنامه‌نویس در خلال کدهای برنامه‌نویسی درج می‌کند تا درک
منطق برنامه را برای دیگر برنامه‌نویسان آسانتر کند (مترجم)
}
ببینید و شروع به نوشتن برنامه خودتان کنید‌.
}

\item{
در حال مهاجرت از
\lr{Dos/Windows}
هستید (‌یا مهاجرت کرده‌اید) و می‌خواهید‌ مجموعه‌ای
از کارها را یکباره انجام دهید.
}

\item{
شما یک نرد
(\lr{nerd})
کامل هستید و هر آموزشی که به دستتان می‌رسد را می‌خوانید.
}
\end{itemize}