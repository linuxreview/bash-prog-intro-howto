\chapter{تابع‌ها}
مثل بیشتر زبان‌های برنامه نویسی‌، می‌توانید از توابع برای گروه‌بندی منطقی قطعه کدها
استفاده کرده و هنر ناب فراخوانی بازگشتی را تمرین کنید.

اعلان یک تابع به سادگی نوشتن
\code{function my\_func \{ my\_code \}‎}
است.

فراخوانی یک تابع درست مثل اجرای یک برنامه است‌، فقط کافی است اسم آن را بنویسید‌.

\section*{چند تابع نمونه}
\begin{latin}
\begin{lstlisting}[numbers=left, numberstyle=\color{red}]
#!/bin/bash
function quit {
	exit
}
function hello {
	echo Hello!
}
hello
quit
echo foo
\end{lstlisting}
\end{latin}

خطوط ۲ تا ۴ محتویات تابع \rlsquote{\code{quit}} و خطوط ۵ تا ۷ محتویات تابع \rlsquote{\code{hello}} هستند. اگر کاملا
مطمئن نیستید که این اسکریپت چه کار می‌کند، لطفا خودتان امتحان کنید.

توجه کنید که نیازی نیست توابع با ترتیب خاصی تعریف شوند.

زمانیکه اسکریپت را اجرا می‌کنید ابتدا تابع \rlsquote{\code{hello}} و سپس تابع \rlsquote{\code{quit}} فراخوانی می‌شود
و برنامه هیچوقت به خط ۱۰ نخواهد رسید
\footnote{به دلیل اینکه اجرای برنامه در خط نهم با فراخوانی تابع \code{quit} خاتمه می‌پذیرد. (مترجم)}.



\section*{توابع نمونه با پارامتر}
\begin{latin}
\begin{lstlisting}
#!/bin/bash
function quit {
	exit
}
function e {
	echo $1
}
e Hello
e World
quit
echo foo
\end{lstlisting}
\end{latin}

این اسکریپت تقریبا شبیه به اسکریپت قبلی است‌. تفاوت اصلی تابع «\code{e}» است‌. این تابع‌،
اولین آرگومان
\footnote{
داده‌ای که در فراخوانی به تابع ارسال می‌شود را «آرگومان» یا «پارامتر» گویند. تفاوت
ظریفی بین دو واژه «آرگومان» و «پارامتر» وجود دارد که ممکن است برای خواننده‌ی علاقه‌مند
جذاب باشد: هر گاه از بیرون تابع به موضوع نگاه شود، داده‌ی ارسال شده به تابع را «پارامتر»
گویند و هرگاه از درون به موضوع نگاه شود داده‌ی وارد شده به تابع را «آرگومان» می‌گویند.
(مترجم)
}
دریافتی‌اش را چاپ می‌کند‌. با آرگومان‌ها‌ی درون توابع‌ درست مانند
آرگومان‌های ارسال شده به اسکریپت رفتار می‌شود.