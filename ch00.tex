\chapter*{مقدمه}
‫حمد و ثناء گوییم خدای متعال را که بار دیگر بر ما منت نهاد تا از وب سایت لینوکس
ریویو برگی سبز به فارسی زبانان جهان و جامعه‌ی آزادمتن ایران تقدیم کرده باشیم.

شکی نیست که گنو/لینوکس موفق‌ترین پروژه‌ی آزادمتن جهان است. امروز گنو/لینوکس در کامپیوترهای
سرور، ابرکامپیوترها، کامپیوترهای رومیزی، گوشی‌های موبایل، دستگاه‌های کپی و بسیاری
از ماشین‌های اطراف ما وجود دارد. متأسفانه با گسترش روز افزون نفوذ این سیستم عامل
در زندگی بشر، کمتر شاهد ترویج فرهنگ آزادی به همراه آن هستیم. گنو/لینوکس تنها یک
سیستم عامل نیست. گنو/لینوکس پرچم‌دار فلسفه‌ی آزادی است. امروز بسیاری «اندروید» و
«اوبونتو» را می‌شناسند ولی گنو/لینوکس که بستر این سیستم عامل‌هاست کمتر به همراه
آن‌ها مطرح شده است. در این شرایط، به خاطر داشتن فرهنگ استفاده از گنو/لینوکس و ترویج
آن بیش از پیش ضروری به نظر می‌رسد.

آنچه در این کتاب عرضه گردیده، آموزش زبان برنامه‌نویسی
\lr{bash}
 می‌باشد. این زبان که
مستقیما با خط فرمان گنو/لینوکس درگیر می‌شود اجازه می‌دهد تا برنامه‌هایی منعطف، کارآمد
و در عین حال ساده و سبک ایجاد کرد. خط فرمان گنو/لینوکس یکی از نقاط قوت غیرقابل انکار
آن می‌باشد که به دلیل ظاهر آن کمتر مورد توجه مخاطبان در بازار کامپیوترهای رومیزی
قرار می‌گیرد. به محض یادگیری نحوه‌ی کار با ترمینال، با مطالعه همین کتاب که خواندن
آن شاید چند ساعتی زمان بگیرد و البته کمی تمرین، می‌توانید برنامه‌‌های دلخواه خودتان
را بنویسید. برنامه‌های این زبان در واقع نیازی به کامپایل شدن ندارند و اگر خطایی
در روند اجرای برنامه باشد، می‌توان به سادگی آن را برطرف نمود.

در ترجمه این کتاب تلاش بر ارائه‌ی ترجمه‌ای مخاطب پسند بوده است؛ در مواردی که از
اصطلاحات تخصصی استفاده شده بود، پانویس مناسب درج گردیده است و مواردی که مبهم بودند
با تلاش مترجمان با توضیحات اضافه‌تر همراه گردید.

در اینجا لازم می‌دانیم از دوستان و یارانی که ما را در تهیه این اثر یاری نمودند تشکر
و قدردانی ویژه بنماییم:

مترجمان: علی قنواتیان، شاهین آزاد، مسعود آموزگار، سلمان محمدی

\vspace*{20pt}
خالی از نقص بودن این اثر ادعای بزرگی است، ولی با غرور اعلام می‌کنیم که نهایت تلاشمان
را برای زدودن اشکالات و ابهامات اثر نموده‌ایم. اشکالات تایپی، خطاهای نگارشی و یا
ترجمه‌های ناصحیح و یا کمتر صحیح و در یک کلام انتقادها و پیشنهادهای خود را با ما
در میان بگذارید تا اثری که به جامعه‌ی آزاد-متن پارسی زبان عرضه می‌شود، سزاوارتر
باشد.

\begin{flushleft}
\textbf{
به نمایندگی از طرف اعضای تیم
\lr{LR}\footnote{\url{http://LinuxReview.ir}} \\
علی قنواتیان
}
\end{flushleft}