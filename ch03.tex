\chapter{همه‌چیز در مورد تغییر مسیر}
\section*{مبحث و مرجع}
سه نوع شرح فایل وجود دارد‌.
\lr{stdin}‏، \lr{stdout} و \lr{stderr}
(سرحرف \lr{STD} = استاندارد).

اساسا می‌توانید‌ اعمال زیر را انجام دهید:
\begin{itemize}
\item{
هدایت \lr{stdout} به یک فایل
}

\item{
هدایت stderr به یک فایل
}

\item{
هدایت \lr{stdout} به \lr{stderr}
}

\item{
هدایت \lr{stderr} به \lr{stdout}
}

\item{
هدایت \lr{stderr} و \lr{stdout} به یک فایل
}

\item{
هدایت \lr{stdout} و \lr{stderr} به \lr{stdout}
}

\item{
هدایت \lr{stderr} و \lr{stdout} به \lr{stderr}
}
\end{itemize}

عدد 1 معرف جریان خروجی استاندارد (\lr{stdout}) و عدد 2 معرف جریان خطا (\lr{stderr}) است.

یک یادداشت کوچک برای مشاهده نحوه عملکرد این‌ها‌: با استفاده از دستور \code{less} شما می‌توانید
هر دو جریان خروجی استاندارد \lr{stdout} (که روی بافر خواهد ماند‌) و جریان خطا (\lr{stderr})
که روی مانیتور نوشته می‌شود را مشاهده کنید‌، ولی جریان \lr{stderr} را نمی‌توانید پیمایش
کنید
\footnote{
برای درک بهتر این تفاوت‌ها یکبار خروجی دستور
\dbquote{\code{‎‎cat /etc/fstab}}
 و یکبار خروجی دستور
\dbquote{\code{ifconfig -\hspace{0.1pt}-help}}
 را به دستور
\code{less}
  هدایت کنید. (مترجم)
}.

\begin{example}{هدایت \lr{stdout} به فایل}
این دستور، خروجی (\lr{‌‌stdout}) یک برنامه را در یک فایل می‌نویسد‌.

\begin{latin}
\begin{lstlisting}
ls -l > ls-l.txt
\end{lstlisting}
\end{latin}

در اینجا یک فایل به اسم «‌\code{ls-l.txt}‌» ساخته می‌شود و محتوای آن چیزی خواهد بود که
هنگام اجرای دستور \code{ls -l} بر روی صفحه‌ی مانیتور مشاهده می‌کنید‌.
\end{example}

\begin{example}{هدایت \lr{stderr} به فایل}
این دستور باعث می‌شود که خروجی خطا (\lr{stderr}) یک برنامه در یک فایل نوشته شود‌.

\begin{latin}
\begin{lstlisting}
grep da * 2> grep-errors.txt
\end{lstlisting}
\end{latin}

در اینجا‌، فایلی موسوم به «\code{grep-errors.txt}» ساخته خواهد شد که در بر دارنده‌ی چیزی
است که در بخش \lr{stderr} خروجی دستور
«‌\code{grep da *}»
می‌بینید
\footnote{
در حالت عادی یک برنامه هر دو خروجی استاندارد و خروجی خطا را بر روی صفحه نمایش نشان
می‌دهد، با کمک
\dbquote{\code{2‎>‎}}
 در این دستور تنها بخشی از خروجی در فایل ذخیره می‌شود که حاوی
پیغام خطا بوده است.
}.
\end{example}

\begin{example}{هدایت \lr{stdout} به \lr{stderr}}
این باعث می‌شود خروجی استاندارد
(\lr{stdout}) 
یک برنامه در شرح فایل
(\lr{file descriptor})
خطا
(\lr{stderr}) نوشته شود.

\begin{latin}
\begin{lstlisting}
grep da * 1>&2
\end{lstlisting}
\end{latin}

در اینجا‌، بخش \lr{stdout} از دستور، به \lr{stderr} ارسال می‌شود. شما ممکن است راه‌های دیگری برای
انجام این کار بشناسید‌.
\end{example}

\begin{example}{هدایت \lr{stderr} به \lr{stdout}}
این باعث می‌شود خروجی خطا (\lr{stderr}) یک برنامه در شرح فایل خروجی استاندارد (\lr{stdout})
نوشته شود.

\begin{latin}
\begin{lstlisting}
grep * 2>&1
\end{lstlisting}
\end{latin}

در این‌جا‌، بخش \lr{stderr} دستور به \lr{stdout‌} ارسال می‌شود. اگر خروجی را به \code{less} لوله‌کشی
کنید، شاهد خطوطی خواهید بود‌ که در حالت عادی «ناپدید می‌شدند». (دلیل این امر وجود
\dbquote{\code{‎2>\&1}}
 است، که خروجی خطا را به خروجی استاندارد هدایت می‌کند. خروجی خطای این دستور
در حالت عادی نمایش داده نمی‌شود. -مترجم-)
\end{example}

\begin{example}{هدایت \lr{stderr} و \lr{stdout} به فایل}
این تمامی خروجی‌های یک برنامه را در یک فایل جای خواهد داد‌. این برای بعضی‌ دستورات
\lr{cron}\footnote{
\lr{cron} سرویسی در گنو/لینوکس است که برای اجرای دستورات در زمان مشخص به کار می‌رود.
دستوراتی که در جدول آن ثبت شده باشند در زمان تعیین شده بر روی سیستم اجرا می‌شوند.
(مترجم)
}
مناسب‌ است‌، وقتی بخواهید دستوری در سکوت کامل اجرا شود‌.

\begin{latin}
\begin{lstlisting}
rm -f $(find / -name core) &> /dev/null
\end{lstlisting}
\end{latin}

این دستور (که می‌تواند در لیست وظایف \lr{cron} باشد) هر فایلی به نام «\code{core}» در هر دایرکتوری‌ای‌
از سیستم که باشد را حذف خواهد کرد. توجه کنید که پیش از پاک کردن خروجی یک دستور باید
بدانید که آن دستور چه کاری انجام می‌دهد.
\end{example}