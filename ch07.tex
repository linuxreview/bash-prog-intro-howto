\chapter{حلقه‌های {\codefont for} و {\codefont while} و {\codefont until}}
در این بخش شما با حلقه‌های \code{for‏}، \code{while} و \code{until} آشنا خواهید شد‌.

حلقه‌ی \textbf{\code{for}} در اینجا کمی با دیگر زبان‌های برنامه نویسی متفاوت است‌. اساساً، این حلقه
به شما اجازه می‌دهد که «لغات» یک رشته متنی را پیمایش کنید
\footnote{
برای درک عملکرد این حلقه تصور کنید یک رشته متنی داشته باشید و بخواهید یک به یک واژه‌های
آن را بررسی کنید. (مترجم)
}.

حلقه‌ی \textbf{\code{while}} تا زمانی که عبارت کنترلی درست باشد‌، قطعه کدی را اجرا می‌کند، و فقط
زمانی متوقف می‌شود که عبارت کنترلی غلط شود (یا این که در روند اجرای تکه کد صراحتاً
با عبارت \code{break} مواجه شود).

حلقه \textbf{\code{until}} تقریباً با حلقه \code{while} برابر است‌، جز این که در این حلقه تا زمانی که عبارت
کنترلی غلط باشد، کد اجرا می‌شود.

اگر فکر می‌کنید‌ \code{while} و \code{until} خیلی به هم شبیه‌اند‌، درست فکر می‌کنید‌.
\section*{برای نمونه}
\begin{latin}
\begin{lstlisting}[numbers=left, numberstyle=\color{red}]
#!/bin/bash
for i in $( ls ); do
	echo item: $i
done
\end{lstlisting}
\end{latin}
\quad \ \ \ در خط دوم ما \code{i} را به عنوان متغیر معرفی کرده‌ایم که می‌تواند مقادیر مختلف حاصل از
اجرای ‎\code{\textdollar( ls )}‎ را بگیرد.

خط سوم در صورت نیاز می‌تواند طولانی‌تر باشد و یا می‌توان خطوط بیشتری پیش از خط \code{done}
اضافه کرد.

\rlsquote{\code{done}} بیانگر این است که استفاده از متغییر \code{\textdollar i} به پایان رسیده و \code{\textdollar ‎i} می‌تواند مقدار
جدیدی دریافت کند.

این اسکریپت چندان مفید نیست، یک راه برای استفاده مفید‌تر از حلقه \code{for} در این مثال
این ‌است که از آن برای دسترسی به فایل‌های خاصی استفاده شود.

\section*{{\codefont for} شبیه {\codefont C}}
\lr{fiesh} پیشنهاد داد که این فرم حلقه‌ را هم اضافه کنیم. این حلقه \code{for} بیشتر شبیه حلقه‌های
\code{for} در زبان‌های \code{C‏}، \code{perl} و \ldots است.
\begin{latin}
\begin{lstlisting}
#!/bin/bash
for i in `seq 1 10`;
do
	echo $i
done
\end{lstlisting}
\end{latin}

\section*{نمونه {\codefont While}}
\begin{latin}
\begin{lstlisting}
#!/bin/bash
COUNTER=0
while [ $COUNTER -lt 10 ]; do
	echo The counter is $COUNTER
	let COUNTER=COUNTER+1
done
\end{lstlisting}
\end{latin}
این اسکریپت ساختار \code{for} زبان‌های مشهور (\code{C}, \code{Pascal}, \code{perl}, و غیره) را سرمشق قرار داده
است.

\section*{نمونه {\codefont until}}
\begin{latin}
\begin{lstlisting}
#!/bin/bash
COUNTER=20
until [ $COUNTER -lt 10 ]; do
	echo COUNTER $COUNTER
	let COUNTER-=1
done
\end{lstlisting}
\end{latin}