\chapter{متفرقه}
\section*{خواندن ورودی‌ کاربر با {\codefont read}}
در بیشتر مواقع شما نیاز دارید که از کاربر ورودی‌هایی دریافت کنید‌، و راه‌های زیادی
برای تحقق این عمل وجود دارد‌. این راه یکی از آن‌هاست‌:
\begin{latin}
\begin{lstlisting}
#!/bin/bash
echo Please, enter your name
read NAME
echo "Hi $NAME!"
\end{lstlisting}
\end{latin}
به روشی دیگر‌، می‌توانید با \code{read} چند مقدار دریافت کنید‌، این مثال بیانگر این روش
است.
\begin{latin}
\begin{lstlisting}
#!/bin/bash
echo Please, enter your firstname and lastname
read FN LN
echo "Hi! $LN, $FN !"
\end{lstlisting}
\end{latin}

\section*{محاسبات ریاضی}
در خط فرمان (یا شل) این را امتحان کنید‌:
\begin{latin}
\begin{lstlisting}
echo 1 + 1
\end{lstlisting}
\end{latin}
اگر انتظار دارید که «\code{2}» را ببینید‌، نا امید می‌شوید‌. اگر می‌خواهید که \lr{bash} اعداد
شما را محاسبه کند‌، چاره این است‌:
\begin{latin}
\begin{lstlisting}
echo $((1+1))
\end{lstlisting}
\end{latin}
این، خروجی منطقی‌تری بدست می‌دهد. این روشی برای محاسبه یک عبارت ریاضیاتی است‌. شما
همچنین به این طریق هم می‌توانید آن را انجام دهید‌:
\begin{latin}
\begin{lstlisting}
echo $[1+1]
\end{lstlisting}
\end{latin}
اگر نیاز به استفاده از کسرها یا محاسبه‌ی عبارات ریاضی بیشتری دارید، می‌توانید از
\code{bc} برای انجام آن محاسبات استفاده کنید‌.

اگر
«\code{echo \textdollar [3/4]}»
را در خط فرمان اجرا کنم‌، عدد صفر برگردانده می‌شود برای این که
\lr{bash} موقع جواب دادن فقط از \rlsquote{اعداد صحیح} استفاده می‌کند‌. اگر 
«\code{echo 3/4|bc -l}»
 را اجرا کنید‌، جواب \code{0.75} برگشت داده می‌شود.

\section*{یافتن bash}
این بخش در پیامی از طرف مایک ارسال شده است (بخش 'قدردانی' را ببینید): شما همیشه
از \code{‎\#!/bin/bash...} استفاده می‌کنید، اگر امکان دارد مثالی از چگونگی یافتن محل استقرار
\lr{bash} بزنید‌. ارجحترین روش «\code{locate bash}» است‌، اما همه‌ی ماشین‌ها \code{locate} را ندارند‌.
اجرای دستور زیر در مسیر ریشه
\footnote{
در لینوکس (و سیستم‌های یونیکس مبنا) همه‌ی فایل‌ها و دایرکتوری‌ها مستقیماً یا به‌طور
غیرمستقیم زیردایرکتوری ریشه هستند. ریشه در این سیستم‌ها با / نمایش داده می‌شود.
(مترجم)
}
از طرف کاربر \lr{root} هم معمولاً نتیجه می‌دهد‌:
\begin{latin}
\begin{lstlisting}
find ./ -name bash
\end{lstlisting}
\end{latin}
مکان‌های پیشنهادی برای چک کردن:

\begin{latin}
\begin{itemize}
\item{\code{ls -l /bin/bash}}
\item{\code{ls -l /sbin/bash}}
\item{\code{ls -l /usr/local/bin/bash}}
\item{\code{ls -l /usr/bin/bash}}
\item{\code{ls -l /usr/sbin/bash}}
\item{\code{ls -l /usr/local/sbin/bash}}
\end{itemize}
\end{latin}
می‌توانید دستور «\code{which bash}» را هم امتحان کنید‌.



\section*{دریافت مقدار بازگردانده شده از یک برنامه}

در بش، مقدار بازگردانده شده از یک برنامه در یک متغییر خاص به نام ‎\code{\textdollar ?}‎ نگه‌داری می‌شود.

در اینجا نحوه دریافت مقدار بازگردانده شده از یک برنامه نشان داده می‌شود. فرض کنید
که دایرکتوری \lr{dada} وجود ندارد (مایک اسم این دایرکتوری را پیشنهاد داد).
\begin{latin}
\begin{lstlisting}
#!/bin/bash
cd /dada &> /dev/null
echo rv: $?
cd $(pwd) &> /dev/null
echo rv: $?
\end{lstlisting}
\end{latin}

\section*{گرفتن خروجی یک دستور}
این اسکریپت کوچک تمام \lr{Table‌}ها از تمام بانک‌های اطلاعاتی را نمایش می‌دهد (با این
فرض که شما \lr{MySQL} را نصب کرده‌ باشید). توجه کنید که دستور «\code{mysql}» را برای استفاده
از نام کاربری و رمز عبور معتبر تغییر دهید.
\begin{latin}
\begin{lstlisting}
#!/bin/bash
DBS=`mysql -uroot -e"show databases"`
for b in $DBS ;
do
	mysql -uroot -e"show tables from $b"
done
\end{lstlisting}
\end{latin}

\section*{چند فایل‌ منبع}

می‌توانید با دستور \code{source} از چند فایل‌ منبع استفاده کنید.

\lr{\_\_TO-DO\_\_}
