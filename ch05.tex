\chapter{متغیر‌ها}
می‌توانید از متغیر‌ها مانند همتاهایشان در تمام زبان‌های برنامه‌نویسی استفاده کنید‌.
در \lr{bash} هیچ‌گونه نوع داده‌ای وجود ندارد‌. یک متغیر در \lr{bash} می‌تواند شامل یک عدد‌،
یک کاراکتر‌ و یا یک رشته از کاراکتر‌ها باشد.

شما نیازی به اعلان متغیر‌ ندارید‌، اختصاص یک مقدار به مرجع آن، متغیر مورد نظر را
می‌سازد.

%\section[مقایسه متغیرها در دو زبان برنامه‌نویسی]
\section*
{مقایسه متغیرها در دو زبان برنامه‌نویسی
\footnote{این بخش توسط مترجم اضافه شده است.}
}
برای درک بهتر موضوع، به دو قطعه کد زیر توجه کنید. این دو قطعه کد ایجاد متغیر در
دو زبان برنامه‌نویسی
\lr{C++}
  و
\lr{bash}
    را نمایش می‌دهد.

تعریف دو متغیر (یکی از نوع عددی و دیگری از نوع رشته‌ای) در \lr{C++}‎:

\begin{latin}
\begin{lstlisting}[language=C++]
int a = 1;
string sta = "bash";
\end{lstlisting}
\end{latin}

تعریف همان دو متغیر در \lr{bash}:

\begin{latin}
\begin{lstlisting}
a=1
sta="bash"
\end{lstlisting}
\end{latin}

\begin{example}{استفاده از متغیرها در اسکریپت سلام دنیا!}
\begin{latin}
\begin{lstlisting}
#!/bin/bash
STR="Hello World!"
echo $STR
\end{lstlisting}
\end{latin}

خط دوم یک متغیر به اسم \code{STR} می‌سازد و رشته
\footnote{
رشته، متن یا دنباله کاراکتری، معادل‌های کلمه‌ی \lr{String} هستند که در زبان‌های برنامه‌نویسی
برای معرفی مجموعه‌ای از کاراکترها (شامل حروف، اعداد و یا هر کاراکتر دیگری) به کار
می‌رود. معمولاً برای اینکه ابتدا و انتهای یک عبارت رشته‌ای مشخص باشد آن را بین دو
علامت کوتیشن " قرار می‌دهند.

}
\dbquote{\code{Hello World!‎}}
 را به آن تخصیص می‌دهد.
در خط سوم مقدار متغییر با گذاشتن نشان \code{\textdollar} در ابتدای اسمش، فراخوانی می‌شود. لطفا توجه
کنید (امتحان کنید!) که اگر از علامت \code{\textdollar} استفاده نکرده باشید خروجی برنامه متفاوت خواهد
بود و احتمالاً آن چیزی که شما انتظار دارید نخواهد بود.
\end{example}

\begin{example}{یک اسکریپت پشتیبان‌گیری خیلی ساده (کمی بهتر از قبلی)}
\begin{latin}
\begin{lstlisting}
#!/bin/bash
OF=/var/my-backup-$(date +%Y%m%d).tgz
tar -cZf $OF /home/me/
\end{lstlisting}
\end{latin}

این اسکریپت چیز دیگری هم معرفی می‌کند‌. اول از همه‌، شما باید با ساخت متغیر‌ها و
اختصاص دادن مقادیر به آن‌ها آشنا شده باشید‌. به این عبارت توجه کنید:
«\code{\textdollar (date +\%Y\%m\%d)‎}».
با اجرای اسکریپت متوجه می‌شوید که بش دستور میان پرانتز را اجرا کرده و خروجی آن را
استفاده کرده است.

توجه کنید که در این اسکریپت، نام فایل خروجی در هر روز متفاوت خواهد بود. دلیلش هم
استفاده از سوئیچ
(\code{+\%Y\%m\%d})
  با دستور
\code{date}\footnote{این دستور برای تنظیم و نمایش تاریخ و زمان سیستم به کار می‌رود.}
 است. می‌توانید ‌با عوض کردن فرمت دستور
\code{date} این بخش از نام فایل را عوض کنید.
\end{example}

مثال‌های بیشتر:
\begin{latin}
\begin{lstlisting}
echo ls
echo $(ls)
\end{lstlisting}
\end{latin}

\section*{متغیرهای محلی}
متغیرهای محلی می‌توانند با استفاده از کلمه‌ی کلیدی \code{local} ساخته شوند.
\begin{latin}
\begin{lstlisting}
#!/bin/bash
HELLO=Hello
function hello {
	local HELLO=World
	echo $HELLO
}
echo $HELLO
hello
echo $HELLO
\end{lstlisting}
\end{latin}

این مثال برای نمایش نحوه‌ی استفاده از متغیرهای محلی باید کافی باشد.
