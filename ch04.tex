\chapter{لوله‌ها}
این قسمت به شکلی بسیار ساده و عملی، در مورد چگونگی استفاده از لوله‌ها و اینکه اصلا
چرا به آن‌ها نیاز دارید توضیح خواهد داد.

\section*{آن‌ها چیستند و چرا از آن‌ها استفاده خواهید کرد}
لوله‌ها به شما اجازه می‌دهند که از خروجی یک نرم‌افزار به عنوان ورودی یک نرم افزار
دیگر استفاده کنید‌
\footnote{
در واقع لوله‌کشی (در بعضی متون «لوله‌بندی» ترجمه شده) امکان ارتباط بین برنامه‌های
مختلف را برای شما فراهم می‌کند. (مترجم)
}.

\begin{example}{یک لوله‌ی ساده با \code{sed}}
این یک روش بسیار ساده برای استفاده از لوله‌هاست‌:
\begin{latin}
\begin{lstlisting}
ls -l | sed -e "s/[aeio]/u/g"
\end{lstlisting}
\end{latin}
اتفاقی که در این دستور رخ می‌دهد این است: اول دستور \code{ls -l} اجرا می‌شود‌، سپس خروجی‌اش‌
به جای چاپ شدن‌ بر روی صفحه نمایش، به برنامه‌ی \code{sed} ارسال (اصطلاحاً لوله‌کشی) می‌شود
که در نتیجه این امر خروجی پس از تغییر، نمایش داده می‌شود.
\end{example}

\begin{example}{یک جایگزین برای \code{ls -l *.txt}}
احتمالاً‌، این یک راه سخت‌تر به جای اجرای \code{ls -l *.txt}‌ است‌، اما این مثال برای توضیح
بهتر لوله‌کشی اینجاست، نه برای حل یک مسئله‌ی دشوار لیست‌گیری‌!

\begin{latin}
\begin{lstlisting}
ls -l | grep "\.txt$"
\end{lstlisting}
\end{latin}

در اینجا‌، خروجی برنامه \code{ls -l} به برنامه \code{grep‌} فرستاده می‌شود، جایی که پس از پردازش،
خطوطی چاپ خواهند شد که با الگوی عبارت باقاعده
\footnote{
عبارت باقاعده یا \lr{regular expression} (یا به اختصار \lr{regex}) مکانیزمی است که برای توصیف
متن‌ ابداع شده است. از این مکانیزم در لینوکس، سیستم‌های مبتنی بر یونیکس و بسیاری
از زبان‌های برنامه‌نویسی برای پردازش متن استفاده می‌شود.
}
روبرو مطابقت داشته باشند‌:
\dbquote{\code{‎\textbackslash .txt\textdollar‎}}.
\end{example}