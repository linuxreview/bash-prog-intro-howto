\chapter{رابط کاربری}
\section*{استفاده از {\codefont select} برای ساخت منوهای ساده}
\begin{latin}
\begin{lstlisting}
#!/bin/bash
OPTIONS="Hello Quit"
select opt in $OPTIONS; do
	if [ "$opt" = "Quit" ]; then
		echo done
		exit
	elif [ "$opt" = "Hello" ]; then
		echo Hello World
	else
		clear
		echo bad option
	fi
done
echo foo
\end{lstlisting}
\end{latin}

اگر این اسکریپت را اجرا کنید، خواهید دید که این، رویای یک برنامه‌نویس برای دسترسی
به منو‌های متنی است. احتمالاً متوجه شده‌اید که این ساختار بسیار شبیه به ساختار «\code{for}»
است‌. فقط به جای تکرار شدن به ازای هر کلمه‌ از \code{‎\textdollar OPTIONS}‌،‌ در هر تکرار، از کاربر
ورودی می‌خواهد.

\section*{استفاده از Line Command}
\begin{latin}
\begin{lstlisting}
#!/bin/bash
if [ -z "$1" ]; then
	echo usage: $0 directory
	exit
fi
SRCD=$1
TGTD="/var/backups/"
OF=home-$(date +%Y%m%d).tgz
tar -cZf $TGTD$OF $SRCD
\end{lstlisting}
\end{latin}

عملکرد این اسکریپت باید برای شما روشن باشد‌. عبارت آمده در ساختار شرطی اول بررسی
می‌کند که برنامه یک آرگومان )‎\code{\textdollar 1}) گرفته است یا خیر، در صورتیکه که چنین نباشد یک
پیام کوتاه نمایش داده و خارج می‌شود. مابقی اسکریپت تا اینجا باید برایتان واضح باشد.
