\chapter{اسکریپت‌های خیلی ساده}
این راهنما سعی دارد کاملاً بر اساس مثال، نکاتی را در مورد برنامه نویسی
\lr{‎Shell Script‎}\footnote{
اصطلاحات «خط فرمان»، «شل»، «بش»، «\lr{bash}»‏، «\lr{shell script}» یا «\lr{bash script}» همه یک
مفهوم دارند. (مترجم)
}
در اختیار شما قرار دهد. در این بخش، شما با چند اسکریپت‌ کوچک آشنا می‌شوید. امیدوارم
به شما در درک و یادگیری چند تکنیک کمک کند.


\section*{
اسکریپتِ سنتیِ سلام دنیا
World) (Hello
}

\begin{latin}
\begin{lstlisting}
#!/bin/bash
echo Hello World
\end{lstlisting}
\end{latin}

این اسکریپت فقط از دو خط تشکیل شده است:

اولین خط به سیستم نشان می‌دهد که از چه برنامه‌ای
برای اجرای فایل استفاده کند.

دومین خط، تنها عملیاتی است که توسط این اسکریپت انجام می‌شود، که عبارت \lr{Hello World}
را در محیط ترمینال چاپ می‌کند.

اگر با اخطاری شبیه به‎
\code{./hello.sh: Command not found}
روبرو شدید‌، احتمالا خط اول
«\code{\#!/bin/bash‌}»
اشتباه است‌. دستور
\code{which bash}
 را اجرا کنید تا مقدار درست این آدرس را ببینید‌
\footnote{
نگارنده برای پیدا کردن \lr{bash} به بخش \lr{finding bash} ارجاء داده است که این بخش نوشته
نشد، مترجمان دو دستور
\dbquote{\code{whereis bash}}
و
\dbquote{\code{which bash}}
را پیش روی خود دیدند، با مقایسه
خروجی این دو دستور
\dbquote{\code{which bash}}
برگزیده شد.
}.

\section*{اسکریپت پشتیبان گیری خیلی ساده}
\begin{latin}
\begin{lstlisting}
#!/bin/bash
tar -cZf /var/my-backup.tgz /home/me/
\end{lstlisting}
\end{latin}

در این اسکریپت، به جای چاپ کردن یک پیغام در ترمینال، من یک فایل فشرده
\lr{tar-ball}
از دایرکتوری
\lr{home}
یک کاربر ایجاد کردم. این اسکریپت چندان آماده‌ی استفاده نیست، ولی
در ادامه همین مقاله یک اسکریپت پشتیبان‌گیری کارآمدتر معرفی خواهد شد
\footnote{
برای اجرای این اسکریپت لازم است سطح دسترسی کاربر ریشه (\lr{root}) را داشته باشید (اسکریپت
را با \code{sudo} اجرا کنید). می‌توانید به پیشنهاد «سعید رسولی» خط دوم این اسکریپت را به
\code{‎tar -cZf /tmp/my-backup.tgz \textasciitilde}
تغییر دهید.
}.