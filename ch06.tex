\chapter{شروط}
شروط به شما اجازه می‌دهند تصمیم بگیرید که عملی انجام شود یا نه‌. این تصمیم‌گیری
بر اساس بررسی درستی عبارات انجام می‌شود‌.

\section*{تئوری خشک}
شروط، فرم‌های زیادی دارند‌. پایه‌ای‌ترین فرم‌:
\lr{{\codefont \textbf{if}} expression {\codefont \textbf{then}} statement}
است که در این فرم بخش «\lr{statement}» زمانی اجرا می‌شود که عبارت «\lr{expression}» از نظر منطقی
درست باشد‌. به عنوان مثال «\code{2<1}» یک عبارت غلط است‌، در صورتی که «\code{1<2}» یک عبارت درست
است‌.

شروط فرم‌های دیگری هم دارند‌، مثل:
\lr{{\codefont \textbf{if}} expression {\codefont \textbf{then}} statement1 {\codefont \textbf{else}} statement2}.
در اینجا «\lr{statement1}» زمانی اجرا می‌شود که «\lr{expression}» درست باشد‌‌، در غیر این
صورت «\lr{statement2}» اجرا می‌شود‌.

یک فرم شرطی دیگر این است:
\begin{flushleft}
\lr{{\codefont \textbf{if}} expression1 {\codefont \textbf{then}} statement1 {\codefont \textbf{else if}} expression2 
{\codefont \textbf{then}} statement2 {\codefont \textbf{else}} statement3}
\end{flushleft}

اگر «\lr{expression1}» درست بود «\lr{statement1}» و اگر «\lr{expression2}» صحیح بود «\lr{statement2}»
و در غیر این صورت «\lr{statement3}» اجرا می‌شود. در این فرم فقط یک بخش
\dbquote{\lr{{\codefont \textbf{ELSE IF}} \lrsquote{expression2} {\codefont \textbf{THEN}} \lrsquote{statement2}}}
 اضافه شده است که اجازه می‌دهد «\lr{statement2}» بعد از اینکه عبارت
«\lr{expression2}» صحیح بود، اجرا شود. نحوه‌ی عملکرد مابقی این فرم‌ را می‌توانید خودتان
تصور کنید. (فرم‌های قبلی را ببینید).

کمی در مورد ساختار دستور:

ساختار پایه‌ای «\code{if}» در \lr{bash} به این صورت است‌:
\begin{flushleft}
\noindent \code{];}عبارت\code{if [} \\
\noindent \code{then} \\
کدی که در صورت درست بودن «عبارت» باید اجرا شود\noindent \\
\noindent \code{fi}
\end{flushleft}

\begin{example}{یک مثال ساده به‌صورت \code{if .. then}}
\begin{latin}
\begin{lstlisting}
#!/bin/bash
if [ "foo" = "foo" ]; then
	echo expression evaluated as true
fi
\end{lstlisting}
\end{latin}
اگر عبارت داخل کروشه درست باشد کدی که بعد از \rlsquote{\code{\code{then}}} و قبل از \rlsquote{\code{\code{fi}}} باشد اجرا می‌شود.
\rlsquote{\code{\code{fi}}} نمایانگر پایان بلوک شرطی کد است.
\end{example}

\begin{example}{نمونه‌ای از یک شرط ابتدایی به صورت \code{if .. then .. else}}
\begin{latin}
\begin{lstlisting}
#!/bin/bash
if [ "foo" = "foo" ]; then
	echo expression evaluated as true
else
	echo expression evaluated as false
fi
\end{lstlisting}
\end{latin}
\end{example}

\begin{example}{شروط به همراه متغیر‌ها}
\begin{latin}
\begin{lstlisting}
#!/bin/bash
T1="foo"
T2="bar"
if [ "$T1" = "$T2" ]; then
	echo expression evaluated as true
else
	echo expression evaluated as false
fi
\end{lstlisting}
\end{latin}
\end{example}
